%----------------------------------------------------------------------------------------
%    PACKAGES AND THEMES
%----------------------------------------------------------------------------------------

\documentclass[aspectratio=169,xcolor=dvipsnames]{beamer}
\usetheme{SimpleDarkBlue}

\usepackage{hyperref}
\usepackage{graphicx} % Allows including images
\usepackage{booktabs} % Allows the use of \toprule, \midrule and \bottomrule in tables
\usepackage{bm}
\DeclareMathOperator*{\argmax}{arg\,max}
%----------------------------------------------------------------------------------------
%    TITLE PAGE
%----------------------------------------------------------------------------------------

\title{Revisiting \texttt{ASkewSGD}: \\New Theoretical Guarantees for QNN Optimization}
\subtitle{Undergraduate Summer Research Internship (2025)}

\author{Presenter: WONG, Hok Fong}

\institute
{
    Department of Computer Science and Engineering \\
    The Chinese University of Hong Kong % Your institution for the title page
}
\date{\today} % Date, can be changed to a custom date

%----------------------------------------------------------------------------------------
%    PRESENTATION SLIDES
%----------------------------------------------------------------------------------------

\begin{document}

\begin{frame}
    % Print the title page as the first slide
    \titlepage
\end{frame}

\begin{frame}{Overview}
    % Throughout your presentation, if you choose to use \section{} and \subsection{} commands, these will automatically be printed on this slide as an overview of your presentation
    \tableofcontents
\end{frame}

\section{Background}
\begin{frame}{Background}

    Modern neural architectures can be up to billions of parameters...

    This induces problems related to...

    \begin{itemize}
        \item the available memory
        \item the slow inference time
    \end{itemize}
\end{frame}

\begin{frame}{Background}

    How to speed up the process? \textit{Ans. Energy-Efficient Computing comes into play}.

    \begin{enumerate}
        \item Knowledge distillation;
        \item Microarchitecture search;
        \item Hardware \& NN-architecture co-design;
        \item Quantization;
        \item Pruning;
        \item Low-rank decomposition.
    \end{enumerate}


    Quantization generally performs better than pruning.\footnote{Andrey Kuzmin, Markus Nagel, Mart van Baalen, Arash Behboodi, Tijkmen Blankevoort. Pruning vs. Quantization: Which is Better? In \texttt{Advances in Neural Information Processing Systems}, 2023. }
\end{frame}

\begin{frame}{Problem Formulation}
    We are interested in solving the optimization problem related to learning a quantized neural network (QNN).

    \begin{block}{The Ultimate Target}
        \[\min_{\mathbf{w}\in\mathcal{Q}} \ell(\mathbf{w})\text{, where }\ell(\mathbf{w})=\mathbb{E}_{(\mathbf{x},y)\sim p_{\text{data}}} [\ell(f(\mathbf{x},\mathbf{w}), y)].\]
        \begin{itemize}
            \item $\ell:\mathbb{R}^d\to \mathbb{R}$ is the training loss;
            \item $\mathcal{Q}$ is the set of quantized weights, e.g. $\{\pm 1\}^d$ for binary quantization;
            \item $d$ is the number of parameters in the neural network;
            \item $f(\mathbf{x},\mathbf{w})$ is the output of the neural network with input $\mathbf{x}$ and weights $\mathbf{w}$;
            \item $p_{\text{data}}$ is the data distribution.
        \end{itemize}
    \end{block}

    This is a non-convex, combinatorial optimization problem.
\end{frame}

\begin{frame}{Problem Formulation}
    \begin{alertblock}{Modified Goal}
        Given a continuously differentiable and reasonably smooth function $\ell$, develop an algorithm
        with certain guarantees that converges to a close-by quantized weight of a locally, or globally, optimal
        continuous weight with similar performance.
    \end{alertblock}



\end{frame}

% \begin{frame}{Blocks of Highlighted Text}
%     In this slide, some important text will be \alert{highlighted} because it's important. Please, don't abuse it.

%     \begin{block}{Block}
%         Sample text
%     \end{block}

%     \begin{alertblock}{Alertblock}
%         Sample text in red box
%     \end{alertblock}

%     \begin{examples}
%         Sample text in green box. The title of the block is ``Examples".
%     \end{examples}
% \end{frame}

% %------------------------------------------------

% \begin{frame}{Multiple Columns}
%     \begin{columns}[c] % The "c" option specifies centered vertical alignment while the "t" option is used for top vertical alignment

%         \column{.45\textwidth} % Left column and width
%         \textbf{Heading}
%         \begin{enumerate}
%             \item Statement
%             \item Derivation
%             \item Example
%         \end{enumerate}

%         \column{.45\textwidth} % Right column and width
%         Lorem ipsum dolor sit amet, consectetur adipiscing elit. Integer lectus nisl, ultricies in feugiat rutrum, porttitor sit amet augue. Aliquam ut tortor mauris. Sed volutpat ante purus, quis accumsan dolor.

%     \end{columns}
% \end{frame}

% %------------------------------------------------
% \section{Second Section}
% %------------------------------------------------

% \begin{frame}{Table}
%     \begin{table}
%         \begin{tabular}{l l l}
%             \toprule
%             \textbf{Treatments} & \textbf{Response 1} & \textbf{Response 2} \\
%             \midrule
%             Treatment 1         & 0.0003262           & 0.562               \\
%             Treatment 2         & 0.0015681           & 0.910               \\
%             Treatment 3         & 0.0009271           & 0.296               \\
%             \bottomrule
%         \end{tabular}
%         \caption{Table caption}
%     \end{table}
% \end{frame}

% %------------------------------------------------

% \begin{frame}{Theorem}
%     \begin{theorem}[Mass--energy equivalence]
%         $E = mc^2$
%     \end{theorem}
% \end{frame}


%------------------------------------------------

% \begin{frame}{References}
%     \footnotesize
%     \bibliography{reference.bib}
%     \bibliographystyle{apalike}
% \end{frame}

%------------------------------------------------


%----------------------------------------------------------------------------------------

\end{document}